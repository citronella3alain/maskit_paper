\documentclass[conference, 12pt, onecolumn]{IEEEtran}
\usepackage{cite}
\usepackage{graphicx}
\usepackage{textcomp}
\usepackage{xcolor}

\date{}

\begin{document}
\title{Maskit: COVID-19 Resiliency through Computer Vision and Robotics}
\author{ \IEEEauthorblockN{Allen Mao}
\IEEEauthorblockA{University of California, Berkeley \\ Berkeley, California \\ \texttt{allenmao@berkeley.edu}}
\and
\IEEEauthorblockN{Hans Gundlach}
\IEEEauthorblockA{University of California, Berkeley \\ Berkeley, California \\ \texttt{11235hans@berkeley.edu}}
}

\maketitle

\begin{abstract}
Abstract goes here\cite{cheng2020role}
\end{abstract}
\section{Introduction}
The COVID-19 pandemic has led to, at the time of this writing, almost half a million deaths, the worst global recession since the Great Depression, and school closures that have affected nearly all of the world's student population. Although reopening efforts have been underway in many parts of the world, their results have been mixed and it is therefore often instead in citizens' own hands to protect themselves from this virus, i.e. social distancing, frequent handwashing, and mask wearing. Although the United States Centers for Disease Control and Prevention (CDC) has issued recommendations for citizens to wear masks to reduce exposure to the virus, the choice to wear a mask has unfortunately evolved to become a political question too. Apart from CDC and World Health Organization (WHO) guidelines to wear facial coverings, i.e. masks, we will not go in depth 
\section{Purpose \& Motivation}
\section{Step by Step Usage Instructions}
\section{Difficulties and Challenges}
\section{Market Evaluation}
\section{Suggested Improvements}
\section{Conclusion}
\bibliographystyle{plain}
\bibliography{citations}


\end{document}
