\documentclass[conference, 12pt, onecolumn]{IEEEtran}
\usepackage{cite}
\usepackage{graphicx}
\usepackage{textcomp}
\usepackage{xcolor}

\date{}

\begin{document}
\title{Maskit: COVID-19 Resiliency through Computer Vision and Robotics}
\author{ \IEEEauthorblockN{Allen Mao}
\IEEEauthorblockA{University of California, Berkeley \\ Berkeley, California \\ \texttt{allenmao@berkeley.edu}}
\and
\IEEEauthorblockN{Hans Gundlach}
\IEEEauthorblockA{University of California, Berkeley \\ Berkeley, California \\ \texttt{11235hans@berkeley.edu}}
}

\maketitle

\section{Introduction}
The COVID-19 pandemic has led to, at the time of this writing, almost half a million deaths, the worst global recession since the Great Depression, and school closures that have affected nearly all of the world's student population. Although reopening efforts have been underway in many parts of the world, their results have been mixed and it is therefore often instead in citizens' own hands to protect themselves from this virus, i.e. social distancing, frequent handwashing, and mask wearing. Although the United States Centers for Disease Control and Prevention (CDC) has issued recommendations for citizens to wear masks to reduce exposure to the virus, the choice to wear a mask has unfortunately evolved to become a political question too. Apart from CDC and World Health Organization (WHO) guidelines to wear facial coverings, i.e. masks, in this article we will not discuss the merits of face coverings; this knowledge is assumed. Instead, we propose, describe, and demonstrate \textit{Maskit}, a computer vision and robotics system that keeps business owners and other ordinary citizens safe by blocking people who fail to wear a mask while letting people who do through entrances.
\subsection{Code Source}
% insert github link here
\section{Purpose \& Motivation}
% insert personal bios and anecdotes here:
% Allen is CS major at UCB and Hans is math + physics. They met at International House. Little did they know that semester would be so abruptly cut off by a pandemic. (Detail brutal dining conditions at I house.) Shock and horror at news coming in about so much death.
% Video of Costco customer refusing to wear a mask. Customers like this are selfish and endanger other people's safety. Even worse, store employees have to personally go escort the person out--who knows how many people this person could infect in that time period? Better, have a computer analyze people automatically: computers can't get infected with COVID and now store employees aren't putting their lives at risk thanks to selfish bass turds.
% This inspired us to think about creating such a system. 
\subsection{Previous Work}
% Computer vision for mask detection
% Computer vision for identifying mask type
\subsection{What is \textit{Maskit}.}
%Google search revealed trained Tensorflow models that recognized human faces. Hans had experience with TF. Allen had experience working with R Pis and servo controls blabla. Combined forces to bring this product together.
%They've come up with prototype that consists of a Rpi and a server. The Rpi is connected to Camera module and servo. It takes pictures at intervals and sends to server as POST request. Server processes pictures with TF model and identifies human faces as masked or unmasked. Then, sends this as response to POST request. If all people identified are wearing a mask, then the raspberry pi activates a welcome flag. Else, people denied entry.
%In model, we have welcome flag, but this can be generalized to any motor, e.g. sliding doors at supermarkets, bolts for regular doors, etc.


\section{Step by Step Usage Instructions}
\subsection{Setting up the server}
\subsection{Running it on R Pi}
\section{Difficulties and Challenges}
\section{Market Evaluation}
\section{Suggested Improvements}
\section{Conclusion}
\bibliographystyle{plain}
\bibliography{citations}


\end{document}
